\documentclass[a4paper, 12pt]{article}
%% placeholder komandai
\usepackage{color}

%% Lietuviški rašmenys, lietuviškų žodžių skirstymas į skiemenis, lietuviški kai kurio dokumento dalių pavadinimai.
\usepackage{polyglossia}
\setdefaultlanguage{lithuanian}

%%paraštes
\usepackage[margin = 2cm]{geometry}
%% pavojingos eilutes
\widowpenalty 1000
\clubpenalty 1000

\usepackage{setspace}
\onehalfspacing


%% Atitraukti pirmą pastraipą.
\usepackage{indentfirst}

%% komandos
\newcommand{\OK}{Oleg Koldun}
\newcommand{\UC}{Ugnė Laima Čižiutė}
\newcommand{\RK}{Rytis Karpuška}
\newcommand{\DK}{Donatas Kučinskas}
\newcommand{\KJ}{Karolis Jocevičius}
\newcommand{\placeholder}{\textbf{\textsf{\textcolor{red}{\fbox{PLACEHOLDER}}}}}


%	instrukcija:
%	"%" 	- komentaras uz jo galima rasyti viska
%	"\" 	- sintaksinis elementas komandai atskirti
%	"{parametras}" 	- parametras komandai
%			pvz: \section{pavadinimas} - sukurs parafrafa su pavadinimu: "pavadinimas"
%	"\\" 	- perkelti teksta i nauja eilute
%	<dvigubas Enter> 	- nauja pastraipa
%	"\section{name} 	- parafrafas
%	"\subsection{name}"	- poskyris
%	"\newpage"	- naujas puslapis
%	
%	software:	instalinat xelatex, reikiamus paketus jis pats suinstalins bandant padaryti faila(~4GB)
%	make instruction: 	xelatex ReferatasZKS.tex
%

\begin{document}



	%%titulinis


	\begin{titlepage}

		\begin{center}
	
			\large {Vilniaus universitetas
	
					Matematikos ir Informatikos fakultetas}
			
			\vspace*{\fill}
	
		\textsc{\Huge placeholder } \\[0.5cm]
		{\Large placeholder}
	
		\vspace{3cm}
	
			\begin{flushright}
				\begin{minipage}{0.3\textwidth}
				{\large Darbą atliko:} \\
					\KJ \\
					\UC \\
					\OK \\
					\DK \\
					\RK
				\end{minipage}
			\end{flushright}
	
			\vspace{\fill}
	
			{\Large Vilnius, \the\year}
	
		\end{center}
	\end{titlepage}
	
	%%anotacija

\section{Anotacija}
	
		\textbf{Darbą atliko:}
		
		
		%%%%%%%%%%%%%%%%%%%%%%%%%%%%%%%%%%%%%
		\textbf{\KJ}
		\begin{flushleft}
		\hspace*{1.5cm}
		Kontaktai:
			karolis.jocevicius@gmail.com
		\\
		\hspace*{1.5cm}
		Indėlis: \placeholder
		\end{flushleft}
		%%%%%%%%%%%%%%%%%%%%%%%%%%%%%%%%%%%%%
		
		\textbf{\UC}
		\begin{flushleft}
		\hspace*{1.5cm}
		Kontaktai: 
			ugne.ciziute@gmail.com
		\\
		\hspace*{1.5cm}
		Indėlis: \placeholder
		\end{flushleft}
		%%%%%%%%%%%%%%%%%%%%%%%%%%%%%%%%%%%%%
		
		\textbf{\OK}
		\begin{flushleft}
		\hspace*{1.5cm}
		Kontaktai: 
			okoldun@gmail.com 
		\\
		\hspace*{1.5cm}
		Indėlis: \placeholder
		\end{flushleft}
		%%%%%%%%%%%%%%%%%%%%%%%%%%%%%%%%%%%%%		
		\textbf{\DK}
		\begin{flushleft}
		\hspace*{1.5cm}
		Kontaktai: 
			donce.lt@gmail.com 
		\\
		\hspace*{1.5cm}
		Indėlis: \placeholder
		\end{flushleft}
		%%%%%%%%%%%%%%%%%%%%%%%%%%%%%%%%%%%%%
		
		\textbf{\RK}
		\begin{flushleft}
		\hspace*{1.5cm}
		Kontaktai: 
			jauleris@gmail.com
		\\
		\hspace*{1.5cm}
		Indėlis: \placeholder
		\end{flushleft}
	\newpage{}	
	
\section{Turinis}
	%%turinis
	
	\tableofcontents

	%%skyriai		



% nuo cia pradedam rasyti
	
\section{pavyzdžiai}
	\subsection{Smart TV}
		
		Tikslas:
		
		Surasti informaciją internete.\\ 	
		Priemonės: 
		
		Pateiktas paieškos langelis ir virtuali klavietūra paieškos tekstui įvesti.\\ 
		
		Rezultatas:
		Gauti paieškos rezultatai.\\ 
		
		Panaudojimumas:\\
		
		
	\subsection{USB jungtis}
		Tikslas:
		
		prijungti kompiuterinę pelę prie kompiuterio.\\ 	
		Priemonės: 
		
		kompiuterinė pelė ir kompiuteris su laisvu USB lizdu.\\ 		
		Rezultatas:
		nepavyko prijungti kompiuterinės pelės iš pirmo karto\\ 	
		Panaudojamumo projektavimo principas, aktualus šioje situacijoje:
		
		Nuspėjamumas – – naudotojas iškart mato, kad USB kištukas turi būti kišamas į USB lizdą, tačiau jis nežino, kuria puse jį reikia kišti.
			
		Sintezavimas – – pusė, kuria kišamas USB kištukas kiekviename įrenginyje gali skirtis.
			
		Atvaizdis - ant USB kištuko yra nurodyta jo viršutinė dalis, tačiau labai neaiškiai. USB lizdas sufleruoja, kuria puse reikia kišti kištuką, tačiau tai nurodanti žymė būna per daug maža, kad naudotojas ją pastebėtų.
			
%% Čia kažkas ne to :D
		Ši mikrobangų krosnelė yra pilnai nuspėjama, todėl naudotojas gali greitai, be jokio vargo ja pasinaudoti. Daugumai naujo tipo mikrobangų krosnelių reikia naudojimosi instrukcijos, dėl jų įvairiapusio funkcionalumo, tačiau dažniausiai naudotojui tereikia greitai pasišildyti maistą.\\
		Pamąstymai:
		
		Projektuotojai padarė teisingai nurodydami kuria puse reikia kišti USB kištuką, tačiau jie nepagalvojo, kad jų užuominos yra per daug menkos ir retas, kas jas pastebės. Geriausias sprendimas būtų spalvos - ausinių/mikrofono kištukai bei jų lizdai yra nuspalvinti atitinkamomis spalvomis, kad naudotojai jų nesupainiotų, ta pati idėja galėtų būti pritaikyta ir USB jungčiai.	
	\subsection{Google pagrindinis puslapis}
		\placeholder
	\subsection{Android skambučiu nutraukimas}
		\placeholder
	\subsection{Mikrobangų krosnelė}
		Tikslas:
		
		pasišildyti maistą\\ 	
		Priemonės: 
		
		laikmatis ir temperatūrą nustatanti rankenėlė\\ 		
		Rezultatas:
		maistas pašilo per nurodytą laika\\ 	
		Panaudojamumo projektavimo principas, aktualus šioje situacijoje:
		
		Nuspėjamumas – naudotojas iškart mato įrenginio veikimo principą.
			
		Sintezavimas – nustačius laikmatį naudotojas iškart pastebi įrenginio būsenos pasikeitimą
			
		Apibendrinimas – seno tipo bei pigios mikrobangų krosnelės veikia tuo pačiu principu.\\
		Pamąstymai:
			
		Ši mikrobangų krosnelė yra pilnai nuspėjama, todėl naudotojas gali greitai, be jokio vargo ja pasinaudoti. Daugumai naujo tipo mikrobangų krosnelių reikia naudojimosi instrukcijos, dėl jų įvairiapusio funkcionalumo, tačiau dažniausiai naudotojui tereikia greitai pasišildyti maistą.	
	\subsection{Microsoft Comamnd Prompt}
	%% FIXME: Lietuviškos raidės!!!!!!
		Tikslas:
		
		Patogiai naudotis keliais komandiniais langais.\\
		Priemonės:
		
		Kompiuteris, klaviatura, pėle.\\
		Rezultatas:
		
		Kompiuteris ivykdo jam ivestas komandas.\\	
		Atvaizdis:
		
		Langas riboto dydžio (neimanoma nustatyti reikiamo vartotojui dydžio).\\	
		Panaudojamumo projektavimo principas, aktualus šioje situacijoje:
		
		Nuspėjamumas - naudotojas iskarto mato kai jo ivestos komandos vykdomos.
		
		Sintezavimas - ivedus komanda arba tekstus vartotojas iškarto mato rezultatus arba komandos vykdymo pradžia.\\		
		Pamastymai:
		
		Sklandžiai veikiantis produktas, bet nėra pilnai pritaikomas vartotojo patogumui.
	\subsection{Mobili 15min.lt versija}	
		Tikslas:
		
		Peržiurėti staipsnius.\\
		Priemonės:
		
		Išmanusis telefonas.\\
		Rezultatas:
		
		Peržiurėtas straipsnis.\\
		Atvaizdis:
		
		Straipsniai lengvai prieinami, bet grižtant atgal iš straipsnio, vėl permetama i puslapio pradžia.\\
		
		Panaudojamumo projektavimo principas, aktualus šioje situacijoje:
		
		Nuspejamumas - nenuspėjama, nes naudotojas iprates kad paspaudus "Back" migtuką, ji gražins į tą pačią vietą.
		
		Sintezavimas - vartotojas iš karto pasiekia jam norima straipsni, bet kaskart sekančio straipsnio reikia ieškoti iš naujo\\		
		Pamąstymai:
		
		Vartotojams butu paprasčiau naudotis jeigu staipsniu archyvas butu išskaldytas į smulkesnius puslapius.
		\placeholder

	\subsection{Nissan Almera laikrodžio nustatymas}
		Tikslas:
		
		Nustatyti teisingą laiką automobilio laikrodyje.\\
		Priemonės:
		
		Mygtukas spidometro skydelyje, kilometražo/laikrodžio ekranas.\\
		Rezutatas:
		
		Nustatytas teisingas laikas.\\
		Atvaizdis:
		
		Mygtukas skirtas laiko nustatymui atrodantis lygiai taip pat kaip ir kilometražo atstatymo mygtukas,
		ekranas kilometražo bei laiko rodymui.\\
		
		Panaudojamumo projektavimo principas, aktualus šioje situacijoje:
		
		Nuspejamumas - nenuspėjama, nes nėra jokių ženklų rodančių mygtuko paskirtį ar veikimo principą.
		
		Sintezavimas - vartotojas gali nustatyti laiką, tačiau nėra aišku kaip tą padaryti, be
		to nustatinėti laiką vienu mygtuku - nepatogu bei užtrunka per daug laiko.\\		
		Pamąstymai:
		
		Vartotojui butu paprasčiau jei laikrodis būtų nustatinėjamas dvejais mygtukais (minutės, valandos)
		bei būtų aiški mygtuko paskirtis.
		
	\subsection{GMail kortelės}
		Tikslas:
		
		Peržiūrėti kortelėse esančius laiškus\\
		Priemonės:
		
		Gmail gautųjų laiškų langas su kortelėmis\\
		Rezutatas:
		
		Peržiūrėti kortelėse surūšiuoti laiškai.\\
		Atvaizdis:
		
		Kortelių juosta virš laiškų sąrašo leidžia pasirinkti rodomą kortelę.\\
		
		Panaudojamumo projektavimo principas, aktualus šioje situacijoje:
		
		Nuspejamumas - nuspėjama - tai standartinis ir įprastas vartotojo sąsajos elementas.
		
		Sintezavimas - Vartotojas gali peržiūrėti surūšiuotus laiškus ir matyti neperskaitytų laišku kieki kortelėje, tačiau peržiūrėjus laišką
		gražinamas pagrindinis laiškų vaizdas, o kortelių juostoje pradingsta neperskaitytų laiškų lankytoje kortelėje kiekis.\\		
		Pamąstymai:
		
		Vartotojui būtų patogiau, jei peržiūrėjus laišką esantį kortelėje būtų grįžtama į prieš tai lankytą kortelę.
		Taip pat nereikėtų išvalyti neperskaitytų kortelės laiškų rodymo.

	\subsection{Google pagrindinis puslapis}
		Tikslas:

		Rasti sau dominančią informaciją.\\
		Priemonės:

		Tekstinė paieška.\\
		Rezultatas:

		Internete rastos informacijos pateikimas svetainių sąrašo, atitinkančių tekstinę užklausą, principu.\\
		Atvaizdis:

		Įprastas teksto įvedimo laukelis.\\

		Panaudojamumo projektavimo principas, atkualus šioje situacijoje:

		Lankstumas - Google sąsaja turi labai mažai elementų (paieškos funkciją galima atlikti pasinaudojus tik teksto įvedimo laukeliu), bet funkcionalumas nėra apribojamas.
		Pavyzdžiui ieškodami youtube vaizdelio, įvykdę paiešką su susijusia užklausa, mums bus pateikiami youtube video pasiūlymai.
		Taip pat adreso/vietos paieška pateikia žemėlapį, kai kurie skaičiavimai atliekami už mus.
		Nespėjus suvesti visos užklausos, iškarto matome pasiūlymus, kaip galėtume užbaigti savo paiešką - stiprinama dialogo iniciatyva.
		Priedo netyčia padarius klaidą, yra pasiūlomi gramatinių klaidų pataisymai - vykdomas užduočių perkelimas.

		Robastiškumas - Vartotojui atsidarius google paieškos puslapį, rodomas pagrindinis sąsajos elementas tiesiai per vidurį, pačioje matomiausioje vietoje - įgyvendinamas matomumo principas.
		Taip pat sąsaja yra labai dinamiška, pradėjus vesti, iškarto matome sąrašą svetainių, atrinktą pagal tai ką jau suvedėme (net ir tuo atveju, jeigu užklausa dar kol kas nebuvo suvesta iki galo).
		Taigi įvykdomas sistemos atsako principas.

		Pamąstymai:

		Lankstumas ir Robastiškumas google sąsajose išsprendžiamas, pateikiant labai intuityvią sąsają, o išmokstamumas sprendžiamas padarant šią sąsąja itin paprastą naudoti.
		Pasinaudoti google, turbūt yra netgi lengviau, negu paklausti draugo to paties klausimo.
	\subsection{Android skambučio nutraukimas}
		Tikslas:

		Pranešti kodėl vartotojas negali atsiliepti.\\
		Priemonės:

		Specialus, išanksto numatytų žinučių sąrašas, skirtų išsiūsti dar skambučio metu.\\
		Rezultatas:

		Nutraukiamas skambutis ir pranešama kodėl vartotojas negalėjo atsiliepti.\\
		Atvaizdis:

		Brūkštelėjus apačioje išvažiuojantis žinučių sąrašas su iš anksto užpildytu tekstu.\\

		Panaudojamumo projektavimo principas, aktualus šioje situacijoje:

		Išmokstamumas - pasinaudojus šiuo funkcionalumu netgi vieną kartą, be papildomų pastangų yra aišku kaip jį panaudoti dar kartą.
		Šis sąsajos elementas yra gerai integruotas į bendrą android aplinką, nes "ištraukiamų" papildomų nustatymų, ar kitų elementų android sistemoje galima sutikti dažnai.
		Tad įgyvendinami nuspėjamumo, atpažįstamumo, bei darnos principai.

		Robastiškumas - Turint omenyje vartojimo aplinką, mums yra būtina minimizuoti veiksmų, reikalingų pasiekti šiam tikslui, skaičių.
		Todėl užduotis tampa sudėtingesnė.
		Laimei šis sąsajos elementas yra atliekamas dviem prisilietimais prie ekrano - tad sąsaja gerai tinkama užduoties atlikimui.
		Taip pat pasirinkus norimą žinutę iš sąrašo, sistema automatiškai nutraukia skambutį, todėl galime teigti, kad sistemos atsako principas irgi yra įgyvendintas sėkmingai.

		pamąstymai:

		Sąsajos elementas ištiesų suprojektuotas patogiai, ir naudojimas yra intuityvus.
		Mano manymu patobulėti galima tik pastiprinus matomumo principą. 
	
	\subsection{Bankomatų ekranai}
		\textbf{Tikslas.}
		Leisti vartotojui pažiūrėti savo banko sąskaitos pinigų likutį, išsiimti grynųjų pinigų.

		\textbf{Priemonės.}
		Ekranas, skirtas rodyti vartotojo banko informaciją.
		Šonuose - mygtukai, kurių funkcijos rašomos ekrane, mygtukų šonuose.

		\textbf{Rezultatas.}
		Vartotojas gauna prieigą prie savo banko sąskaitos, gali į ją įnešti ir iš jos išsiimti pinigų.

		\textbf{Panaudojamumas.}
		Jeigu nėra saulės, naudotis visai patogu.
		Tačiau jeigu diena saulėta, šviesa nuo ekrano labai atsispindi, todėl įžiūrėti tekstą yra gan sunku.

		\textbf{Pamąstymai.}
		Vartotojui, ypač turinčiam nusilpusį regėjimą, naudotis tokiu interfeisu gali būti sunku.
		Reikėtų kaip nors pagerinti ekrano matumumą - pavyzdžiui, visus bankomatus laikyti patalpų viduje arba įrengti budelę.
		Tokiu būdu bankomato ekranas būtų apsaugotas nuo saulės atspindžių ir naudotis šiuo interfeisu taptų kur kas lengviau.

		
	\subsection{Numatytoji paveikslėlių peržiūra Windows 8 sistemoje}
		\textbf{Tikslas.}
		Pavaizduoti vartotojo atidarytą paveikslėlį ekrane.
		
		\textbf{Priemonės.}
		Paveiksliuko failas, kurį norima atidaryti; monitorius, kuriame bus pavaizduotas paveiksliukas.

		\textbf{Rezultatas.}
		Monitoriaus ekrane rodomas norimas paveiksliukas.

		\textbf{Panaudojamumas.}

		\textbf{Nuspėjamumas.}
		Interfeiso nuspėjamumas nėra prastas - atidaromas paveikslėlis rodomas per visą ekraną.
		\textbf{Sintezavimas.}
		Vartotojo norimas atidaryti paveiksliukas atsidarė, tačiau jis užima visą ekraną - norint pasiekti kokią nors informaciją iš darbastalio, reikia grįžti į darbastalį, o dėl to reikia atlikti papildomų veiksmų bei uždaryti paveiksliuką.

		\textbf{Pamąstymai.}
		Tokia paveiksliuko peržiūra labai nepatogi, jeigu norima peržiūrėti kelis paveikslėlius vienu metu arba atliekant kelis darbus vienu metu.
		Tai ypač nepatogu pasirodo vartotojams, kurie yra pratę prie įprastų paveikliukų peržiūros programų, kurios paveiksliuko per visiškai visą langą iš karto neatidaro (arba bent leidžia sumažinti).

\end{document}
