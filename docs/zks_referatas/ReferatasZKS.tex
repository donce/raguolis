\documentclass[a4paper, 12pt]{article}
%% placeholder komandai
\usepackage{color}

%% Lietuviški rašmenys, lietuviškų žodžių skirstymas į skiemenis, lietuviški kai kurio dokumento dalių pavadinimai.
\usepackage{polyglossia}
\setdefaultlanguage{lithuanian}

%%paraštes
\usepackage[margin = 2cm]{geometry}
%% pavojingos eilutes
\widowpenalty 1000
\clubpenalty 1000

\usepackage{setspace}
\onehalfspacing


%% Atitraukti pirmą pastraipą.
\usepackage{indentfirst}

%% komandos
\newcommand{\OK}{Oleg Koldun}
\newcommand{\UC}{Ugnė Laima Čižiutė}
\newcommand{\RK}{Rytis Karpuška}
\newcommand{\DK}{Donatas Kučinskas}
\newcommand{\KJ}{Karolis Jocevičius}
\newcommand{\placeholder}{\textbf{\textsf{\textcolor{red}{\fbox{PLACEHOLDER}}}}}


%	instrukcija:
%	"%" 	- komentaras uz jo galima rasyti viska
%	"\" 	- sintaksinis elementas komandai atskirti
%	"{parametras}" 	- parametras komandai
%			pvz: \section{pavadinimas} - sukurs parafrafa su pavadinimu: "pavadinimas"
%	"\newline" 	- perkelti teksta i nauja eilute
%	<dvigubas Enter> 	- nauja pastraipa
%	"\section{name} 	- parafrafas
%	"\subsection{name}"	- poskyris
%	"\newpage"	- naujas puslapis
%	
%	software:	instalinat xelatex, reikiamus paketus jis pats suinstalins bandant padaryti faila(~4GB)
%	make instruction: 	xelatex ReferatasZKS.tex
%

\begin{document}



	%%titulinis


	\begin{titlepage}

		\begin{center}
	
			\large {Vilniaus universitetas
	
					Matematikos ir Informatikos fakultetas}
			
			\vspace*{\fill}
	
		\textsc{\Huge placeholder } \\[0.5cm]
		{\Large placeholder}
	
		\vspace{3cm}
	
			\begin{flushright}
				\begin{minipage}{0.3\textwidth}
				{\large Darbą atliko:} \\
					\KJ \\
					\UC \\
					\OK \\
					\DK \\
					\RK
				\end{minipage}
			\end{flushright}
	
			\vspace{\fill}
	
			{\Large Vilnius, \the\year}
	
		\end{center}
	\end{titlepage}
	
	%%anotacija

\section{Anotacija}
	
		\textbf{Darbą atliko:}
		
		
		%%%%%%%%%%%%%%%%%%%%%%%%%%%%%%%%%%%%%
		\textbf{\KJ}
		\begin{flushleft}
		\hspace*{1.5cm}
		Kontaktai:
			karolis.jocevicius@gmail.com
		\newline
		\hspace*{1.5cm}
		Indėlis: \placeholder
		\end{flushleft}
		%%%%%%%%%%%%%%%%%%%%%%%%%%%%%%%%%%%%%
		
		\textbf{\UC}
		\begin{flushleft}
		\hspace*{1.5cm}
		Kontaktai: 
			ugne.ciziute@gmail.com
		\newline
		\hspace*{1.5cm}
		Indėlis: \placeholder
		\end{flushleft}
		%%%%%%%%%%%%%%%%%%%%%%%%%%%%%%%%%%%%%
		
		\textbf{\OK}
		\begin{flushleft}
		\hspace*{1.5cm}
		Kontaktai: 
			okoldun@gmail.com 
		\newline
		\hspace*{1.5cm}
		Indėlis: \placeholder
		\end{flushleft}
		%%%%%%%%%%%%%%%%%%%%%%%%%%%%%%%%%%%%%		
		\textbf{\DK}
		\begin{flushleft}
		\hspace*{1.5cm}
		Kontaktai: 
			donce.lt@gmail.com 
		\newline
		\hspace*{1.5cm}
		Indėlis: \placeholder
		\end{flushleft}
		%%%%%%%%%%%%%%%%%%%%%%%%%%%%%%%%%%%%%
		
		\textbf{\RK}
		\begin{flushleft}
		\hspace*{1.5cm}
		Kontaktai: 
			jauleris@gmail.com
		\newline
		\hspace*{1.5cm}
		Indėlis: \placeholder
		\end{flushleft}
	\newpage{}	
	
\section{Turinis}
	%%turinis
	
	\tableofcontents

	%%skyriai		



% nuo cia pradedam rasyti
	
\section{pavyzdžiai}
	\subsection{Smart TV}\newline 
		
		Tikslas:
		
		Surasti informaciją internete.\newline 	
		Priemonės: 
		
		Pateiktas paieškos langelis ir virtuali klavietūra paieškos tekstui įvesti.\newline 
		
		Rezultatas:
		Gauti paieškos rezultatai.\newline 
		
		Panaudojimumas:\newline
		
		
	\subsection{USB jungtis}
		Tikslas:
		
		prijungti kompiuterinę pelę prie kompiuterio.\newline 	
		Priemonės: 
		
		kompiuterinė pelė ir kompiuteris su laisvu USB lizdu.\newline 		
		Rezultatas:
		nepavyko prijungti kompiuterinės pelės iš pirmo karto\newline 	
		Panaudojamumo projektavimo principas, aktualus šioje situacijoje:
		
		Nuspėjamumas – – naudotojas iškart mato, kad USB kištukas turi būti kišamas į USB lizdą, tačiau jis nežino, kuria puse jį reikia kišti.
			
		Sintezavimas – – pusė, kuria kišamas USB kištukas kiekviename įrenginyje gali skirtis.
			
		Atvaizdis - ant USB kištuko yra nurodyta jo viršutinė dalis, tačiau labai neaiškiai. USB lizdas sufleruoja, kuria puse reikia kišti kištuką, tačiau tai nurodanti žymė būna per daug maža, kad naudotojas ją pastebėtų.
			
%% Čia kažkas ne to :D
		Ši mikrobangų krosnelė yra pilnai nuspėjama, todėl naudotojas gali greitai, be jokio vargo ja pasinaudoti. Daugumai naujo tipo mikrobangų krosnelių reikia naudojimosi instrukcijos, dėl jų įvairiapusio funkcionalumo, tačiau dažniausiai naudotojui tereikia greitai pasišildyti maistą.\newline
		Pamąstymai:
		
		Projektuotojai padarė teisingai nurodydami kuria puse reikia kišti USB kištuką, tačiau jie nepagalvojo, kad jų užuominos yra per daug menkos ir retas, kas jas pastebės. Geriausias sprendimas būtų spalvos - ausinių/mikrofono kištukai bei jų lizdai yra nuspalvinti atitinkamomis spalvomis, kad naudotojai jų nesupainiotų, ta pati idėja galėtų būti pritaikyta ir USB jungčiai.	
	\subsection{Google pagrindinis puslapis}
		\placeholder
	\subsection{Android skambučiu nutraukimas}
		\placeholder
	\subsection{Mikrobangų krosnelė}
		Tikslas:
		
		pasišildyti maistą\newline 	
		Priemonės: 
		
		laikmatis ir temperatūrą nustatanti rankenėlė\newline 		
		Rezultatas:
		maistas pašilo per nurodytą laika\newline 	
		Panaudojamumo projektavimo principas, aktualus šioje situacijoje:
		
		Nuspėjamumas – naudotojas iškart mato įrenginio veikimo principą.
			
		Sintezavimas – nustačius laikmatį naudotojas iškart pastebi įrenginio būsenos pasikeitimą
			
		Apibendrinimas – seno tipo bei pigios mikrobangų krosnelės veikia tuo pačiu principu.\newline
		Pamąstymai:
			
		Ši mikrobangų krosnelė yra pilnai nuspėjama, todėl naudotojas gali greitai, be jokio vargo ja pasinaudoti. Daugumai naujo tipo mikrobangų krosnelių reikia naudojimosi instrukcijos, dėl jų įvairiapusio funkcionalumo, tačiau dažniausiai naudotojui tereikia greitai pasišildyti maistą.	
	\subsection{Microsoft Comamnd Prompt}
	%% FIXME: Lietuviškos raidės!!!!!!
		Tikslas:
		
		Patogiai naudotis keliais komandiniais langais.\newline
		Priemonės:
		
		Kompiuteris, klaviatura, pėle.\newline
		Rezultatas:
		
		Kompiuteris ivykdo jam ivestas komandas.\newline	
		Atvaizdis:
		
		Langas riboto dydžio (neimanoma nustatyti reikiamo vartotojui dydžio).\newline	
		Panaudojamumo projektavimo principas, aktualus šioje situacijoje:
		
		Nuspėjamumas - naudotojas iskarto mato kai jo ivestos komandos vykdomos.
		
		Sintezavimas - ivedus komanda arba tekstus vartotojas iškarto mato rezultatus arba komandos vykdymo pradžia.\newline		
		Pamastymai:
		
		Sklandžiai veikiantis produktas, bet nėra pilnai pritaikomas vartotojo patogumui.
	\subsection{Mobili 15min.lt versija}	
		Tikslas:
		
		Peržiurėti staipsnius.\newline
		Priemonės:
		
		Išmanusis telefonas.\newline
		Rezultatas:
		
		Peržiurėtas straipsnis.\newline
		Atvaizdis:
		
		Straipsniai lengvai prieinami, bet grižtant atgal iš straipsnio, vėl permetama i puslapio pradžia.\newline
		
		Panaudojamumo projektavimo principas, aktualus šioje situacijoje:
		
		Nuspejamumas - nenuspėjama, nes naudotojas iprates kad paspaudus "Back" migtuką, ji gražins į tą pačią vietą.
		
		Sintezavimas - vartotojas iš karto pasiekia jam norima straipsni, bet kaskart sekančio straipsnio reikia ieškoti iš naujo\newline		
		Pamąstymai:
		
		Vartotojams butu paprasčiau naudotis jeigu staipsniu archyvas butu išskaldytas į smulkesnius puslapius.
		\placeholder

	\subsection{Nissan Almera laikrodžio nustatymas}
		Tikslas:
		
		Nustatyti teisingą laiką automobilio laikrodyje.\newline
		Priemonės:
		
		Mygtukas spidometro skydelyje, kilometražo/laikrodžio ekranas.\newline
		Rezutatas:
		
		Nustatytas teisingas laikas.\newline
		Atvaizdis:
		
		Mygtukas skirtas laiko nustatymui atrodantis lygiai taip pat kaip ir kilometražo atstatymo mygtukas,
		ekranas kilometražo bei laiko rodymui.\newline
		
		Panaudojamumo projektavimo principas, aktualus šioje situacijoje:
		
		Nuspejamumas - nenuspėjama, nes nėra jokių ženklų rodančių mygtuko paskirtį ar veikimo principą.
		
		Sintezavimas - vartotojas gali nustatyti laiką, tačiau nėra aišku kaip tą padaryti, be
		to nustatinėti laiką vienu mygtuku - nepatogu bei užtrunka per daug laiko.\newline		
		Pamąstymai:
		
		Vartotojui butu paprasčiau jei laikrodis būtų nustatinėjamas dvejais mygtukais (minutės, valandos)
		bei būtų aiški mygtuko paskirtis.
		
	\subsection{GMail kortelės}
		Tikslas:
		
		Peržiūrėti kortelėse esančius laiškus\newline
		Priemonės:
		
		Gmail gautųjų laiškų langas su kortelėmis\newline
		Rezutatas:
		
		Peržiūrėti kortelėse surūšiuoti laiškai.\newline
		Atvaizdis:
		
		Kortelių juosta virš laiškų sąrašo leidžia pasirinkti rodomą kortelę.\newline
		
		Panaudojamumo projektavimo principas, aktualus šioje situacijoje:
		
		Nuspejamumas - nuspėjama - tai standartinis ir įprastas vartotojo sąsajos elementas.
		
		Sintezavimas - Vartotojas gali peržiūrėti surūšiuotus laiškus ir matyti neperskaitytų laišku kieki kortelėje, tačiau peržiūrėjus laišką
		gražinamas pagrindinis laiškų vaizdas, o kortelių juostoje pradingsta neperskaitytų laiškų lankytoje kortelėje kiekis.\newline		
		Pamąstymai:
		
		Vartotojui būtų patogiau, jei peržiūrėjus laišką esantį kortelėje būtų grįžtama į prieš tai lankytą kortelę.
		Taip pat nereikėtų išvalyti neperskaitytų kortelės laiškų rodymo.

	\subsection{Google pagrindinis puslapis}
		Tikslas:

		Rasti sau dominančią informaciją.\newline
		Priemonės:

		Tekstinė paieška.\newline
		Rezultatas:

		Internete rastos informacijos pateikimas svetainių sąrašo, atitinkančių tekstinę užklausą, principu.\newline
		Atvaizdis:

		Įprastas teksto įvedimo laukelis.\newline

		Panaudojamumo projektavimo principas, atkualus šioje situacijoje:

		Lankstumas - Google sąsaja turi labai mažai elementų (paieškos funkciją galima atlikti pasinaudojus tik teksto įvedimo laukeliu), bet funkcionalumas nėra apribojamas.
		Pavyzdžiui ieškodami youtube vaizdelio, įvykdę paiešką su susijusia užklausa, mums bus pateikiami youtube video pasiūlymai.
		Taip pat adreso/vietos paieška pateikia žemėlapį, kai kurie skaičiavimai atliekami už mus.
		Nespėjus suvesti visos užklausos, iškarto matome pasiūlymus, kaip galėtume užbaigti savo paiešką - stiprinama dialogo iniciatyva.
		Priedo netyčia padarius klaidą, yra pasiūlomi gramatinių klaidų pataisymai - vykdomas užduočių perkelimas.

		Robastiškumas - Vartotojui atsidarius google paieškos puslapį, rodomas pagrindinis sąsajos elementas tiesiai per vidurį, pačioje matomiausioje vietoje - įgyvendinamas matomumo principas.
		Taip pat sąsaja yra labai dinamiška, pradėjus vesti, iškarto matome sąrašą svetainių, atrinktą pagal tai ką jau suvedėme (net ir tuo atveju, jeigu užklausa dar kol kas nebuvo suvesta iki galo).
		Taigi įvykdomas sistemos atsako principas.

		Pamąstymai:

		Lankstumas ir Robastiškumas google sąsajose išsprendžiamas, pateikiant labai intuityvią sąsają, o išmokstamumas sprendžiamas padarant šią sąsąja itin paprastą naudoti.
		Pasinaudoti google, turbūt yra netgi lengviau, negu paklausti draugo to paties klausimo.
	\subsection{Android skambučio nutraukimas}
		Tikslas:

		Pranešti kodėl vartotojas negali atsiliepti.\newline
		Priemonės:

		Specialus, išanksto numatytų žinučių sąrašas, skirtų išsiūsti dar skambučio metu.\newline
		Rezultatas:

		Nutraukiamas skambutis ir pranešama kodėl vartotojas negalėjo atsiliepti.\newline
		Atvaizdis:

		Brūkštelėjus apačioje išvažiuojantis žinučių sąrašas su iš anksto užpildytu tekstu.\newline

		Panaudojamumo projektavimo principas, aktualus šioje situacijoje:

		Išmokstamumas - pasinaudojus šiuo funkcionalumu netgi vieną kartą, be papildomų pastangų yra aišku kaip jį panaudoti dar kartą.
		Šis sąsajos elementas yra gerai integruotas į bendrą android aplinką, nes "ištraukiamų" papildomų nustatymų, ar kitų elementų android sistemoje galima sutikti dažnai.
		Tad įgyvendinami nuspėjamumo, atpažįstamumo, bei darnos principai.

		Robastiškumas - Turint omenyje vartojimo aplinką, mums yra būtina minimizuoti veiksmų, reikalingų pasiekti šiam tikslui, skaičių.
		Todėl užduotis tampa sudėtingesnė.
		Laimei šis sąsajos elementas yra atliekamas dviem prisilietimais prie ekrano - tad sąsaja gerai tinkama užduoties atlikimui.
		Taip pat pasirinkus norimą žinutę iš sąrašo, sistema automatiškai nutraukia skambutį, todėl galime teigti, kad sistemos atsako principas irgi yra įgyvendintas sėkmingai.

		pamąstymai:

		Sąsajos elementas ištiesų suprojektuotas patogiai, ir naudojimas yra intuityvus.
		Mano manymu patobulėti galima tik pastiprinus matomumo principą. 
\end{document}
